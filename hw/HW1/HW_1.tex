\documentclass[12pt,letterpaper]{article}
\usepackage{fullpage}
\usepackage{amsmath, amsfonts, amssymb, amsthm, amscd} % Combined math packages
\usepackage{tikz}     % For drawing matrices
\usepackage{listings, pythontex} % For code snippets
\usepackage{enumerate}
\usepackage{fancyhdr}
\usepackage{mathrsfs}
\usepackage{xcolor}
\usepackage{graphicx}
\usepackage{hyperref}
\usepackage{booktabs}
\usepackage{relsize}

\hypersetup{%
  colorlinks=true,
  linkcolor=blue,
  linkbordercolor={0 0 1}
}

\setlength{\parindent}{0.0in}
\setlength{\parskip}{0.05in}

%%%%%%%%%%%%%%%%%%%%%%%%%%%%%%%%%%%%%%%%%%%%%%%%%%%%%%%%%%

%%%%%%%%%%%%%%%%%%%%%%%% CHANGE HERE %%%%%%%%%%%%%%%%%%%%
\newcommand\course{STA4724}
\newcommand\semester{Spring 2025}
\newcommand\hwnumber{1}                 % <-- ASSIGNMENT #
\newcommand\NetIDa{Albert Jojo}          % <-- YOUR NAME
\newcommand\NetIDb{5390131}              % <-- STUDENT ID #
%%%%%%%%%%%%%%%%%%%%%%%%%%%%%%%%%%%%%%%%%%%%%%%%%%%%%%%%%%

%%%%%%%%%%%%%%%%% DO NOT CHANGE HERE %%%%%%%%%%%%%%%%%%%%
\pagestyle{fancyplain}
\headheight 35pt
\lhead{\NetIDa \\ \NetIDb} % Combined name and ID in the left header
\chead{\textbf{\Large Homework \hwnumber}}
\rhead{\course \\ \semester}
\lfoot{}
\cfoot{}
\rfoot{\small\thepage}
\headsep 1.5em
%%%%%%%%%%%%%%%%%%%%%%%%%%%%%%%%%%%%%%%%%%%%%%%%%%%%%%%%%%

\begin{document}
\subsection*{1a. Variable Types}

\textbf{int}: the variable type that holds signed integer values. (print(8) Output: 
\begin{pycode}
print(8)
\end{pycode}
)

\textbf{string}: a sequence of 0 or more characters that is enclosed with either " or '.(print("Hi") Output: 
\begin{pycode}
print("Hi")
\end{pycode}
)

\textbf{float}: the variable type that is denoted by any rational numbers (print(3.1415) Output: 
\begin{pycode}
print(3.1415)
\end{pycode}
)

\subsection*{1b. Strings vs. Lists}
Strings, compared to lists, are immutable, meaning in Python a string can't be directly modified after initialization. All the string operations create a copy of the original string to modify it for different functions. On the other hand, a list is mutable, meaning that it can be directly modified after initialization. This allows you to extend or append a value to a list.

\subsection*{1c. Example of List Modification}
\begin{pycode}
fruits = ['apple', 'banana', 'cherry', 'persimmon', 'jackfruit']
fruits.append('orange')
print(f"Output: {fruits}")
\end{pycode}

\subsection*{1d. Image Processing Example}

\begin{pycode}
from PIL import Image
import numpy as np

# Load the Image
img = Image.open('image-hw1.png')

# Convert the image to img_matrix
img_matrix = np.array(img)

# Select the slice of the image to change the color to red
img_matrix[:20, :, :] = [255, 0, 0, 255]

# Reconvert the image_matrix to img
img = Image.fromarray(img_matrix)

# Save the modified image
output_filename = "modified_image.png"
img.save(output_filename)

# Optionally display the image (this will not show in the LaTeX PDF)
img.show()
\end{pycode}

\begin{center}
   \includegraphics[width=\textwidth]{modified_image.png}
\end{center}

    \newpage

    2a. Prove \( IA = A \), where \( A \) is an \( n \times l \) matrix and \( I \) is an \( n \times n \) identity matrix:
    \[
        A = \begin{bmatrix}
            a_{11} & \cdots & a_{1l} \\
            a_{n1} & \cdots & a_{nl}
        \end{bmatrix}
    \]
    \paragraph{Proof of \( IA = A \):}
    Let \( A \) be an \( n \times l \) matrix and \( I \) the \( n \times n \) identity matrix. 
    The element in the \( i \)-th row and \( j \)-th column of \( IA \) is:
    \[
    (IA)_{ij} = \sum_{k=1}^{n} I_{ik} A_{kj}.
    \]
    Since \( I_{ik} = 1 \) if \( i = k \) and \( 0 \) otherwise, this simplifies to:
    \[
    (IA)_{ij} = I_{ii} A_{ij} = A_{ij}.
    \]
    Thus, \( IA = A \). \qedhere

    
    
    \vspace{1.2cm}

    2b. Let \( A \) be an \( n \times n \) square matrix. Prove that \( A \) and \( A^T \) have exactly the same eigenvalues.
    \paragraph{Proof \( A \) and \( A^T \) have the same eigenvalues:}
    The eigenvalues of \( A \) are solutions to \( \det(A - \lambda I) = 0 \). For \( A^T \):
    \[
    \det(A^T - \lambda I) = \det((A - \lambda I)^T) = \det(A - \lambda I).
    \]
    Since \( \det(A^T - \lambda I) = \det(A - \lambda I) \), \( A \) and \( A^T \) share the same characteristic equation, hence the same eigenvalues. \qedhere

    \vspace{1cm}
    2c. Let A be an n × n square matrix. Prove $A^{-1}$ exists if and only if 0 is not an eigenvalue of A

    \paragraph{Proof}  
    \textbf{1st:} Assume \( A^{-1} \) exists. Then \( \det(A) \neq 0 \).  
    If \( \lambda = 0 \) were an eigenvalue, then \( \det(A - 0 \cdot I) = \det(A) = 0 \), contradicting \( \det(A) \neq 0 \).  
    Thus, \( 0 \) is not an eigenvalue.  
    
    \textbf{2nd:} Assume \( 0 \) is not an eigenvalue. Then \( \det(A) = \det(A - 0 \cdot I) \neq 0 \).  
    By the invertible matrix theorem, \( A \) is invertible.  
    
    \vspace{0.2cm}  
    \noindent
    \boxed{\text{\( A^{-1} \) exists if and only if \( 0 \) is not an eigenvalue of \( A \).}}

    \newpage

    2d. Let \( A \) be an \( n \times n \) square matrix such that \( A^{-1} \) exists. Prove that if \( \lambda \) is an eigenvalue of \( A \), then \( \frac{1}{\lambda} \) is an eigenvalue of \( A^{-1} \).

    \paragraph{Proof}
    Let \( \mathbf{x} \neq \mathbf{0} \) be an eigenvector of \( A \) with eigenvalue \( \lambda \). Then:
    \[
    A\mathbf{x} = \lambda\mathbf{x}.
    \]
    Since \( A \) is invertible, \( \lambda \neq 0 \). Multiply both sides by \( A^{-1} \):
    \[
    A^{-1}(A\mathbf{x}) = A^{-1}(\lambda\mathbf{x}) \implies \mathbf{x} = \lambda A^{-1}\mathbf{x}.
    \]
    Dividing by \( \lambda \):
    \[
    A^{-1}\mathbf{x} = \frac{1}{\lambda}\mathbf{x}.
    \]
    Thus, \( \frac{1}{\lambda} \) is an eigenvalue of \( A^{-1} \). \qed

    \vspace{1.2cm}

    2e. Prove that if \(A^2 = I\), eigenvalues of A must be 1 or -1

    \paragraph{Proof}
    Consider \(\lambda, x\)

    \begin{align*}
        Ax &= \lambda x \\
        A^2x &= \lambda Ax  \\
        Ix &= \lambda Ax  \\
        x &= \lambda Ax  \\
        x &= \lambda (\lambda x)  \\
        x &= \lambda^2 x \\
        \implies \lambda^2 &= 1 \\
        \therefore \lambda &= \pm 1
    \end{align*}

    \newpage

    2f. Find whether the matrix A = \(
        \begin{bmatrix}
            1 & 2 \\
            0 & -1 
        \end{bmatrix}
    \) has any defective eigenvalues. 

 
\paragraph{Solution:}

\subparagraph{Find eigenvalues (algebraic multiplicity, \( t \))}
\[
\det(A - \lambda I) = \begin{vmatrix}
1 - \lambda & 2 \\
0 & -1 - \lambda
\end{vmatrix} = (1 - \lambda)(-1 - \lambda) - (0 \cdot 2).
\]
\[
(1 - \lambda)(-1 - \lambda) = -(1 - \lambda)(1 + \lambda) = -1 + \lambda^2.
\]
\[
\lambda^2 - 1 = 0 \implies \lambda^2 = 1 \implies \lambda = \pm 1.
\]
Algebraic multiplicities:
\[
\lambda_1 = 1 \quad (t_1 = 1), \quad \lambda_2 = -1 \quad (t_2 = 1).
\]

\subparagraph{Find geometric multiplicities (\( g \))}

\textbf{For \( \lambda_1 = 1 \):}
\[
A - I = \begin{bmatrix}
0 & 2 \\
0 & -2
\end{bmatrix}.
\]
Solve \( (A - I)\mathbf{x} = \mathbf{0} \):
\[
0x + 2y = 0 \implies y = 0.
\]
Eigenvectors: \( \mathbf{x} = x \begin{bmatrix} 1 \\ 0 \end{bmatrix} \). \\
Geometric multiplicity: \( g_1 = 1 \).

\textbf{For \( \lambda_2 = -1 \):}
\[
A + I = \begin{bmatrix}
2 & 2 \\
0 & 0
\end{bmatrix}.
\]
Solve \( (A + I)\mathbf{x} = \mathbf{0} \):
\[
2x + 2y = 0 \implies x = -y.
\]
Eigenvectors: \( \mathbf{x} = y \begin{bmatrix} -1 \\ 1 \end{bmatrix} \). \\
Geometric multiplicity: \( g_2 = 1 \).

\subparagraph{Check for defective eigenvalues} 

A defective eigenvalue occurs if \( g < t \). In this case
\[
\begin{cases}
g_1 = t_1 = 1 \implies \lambda_1 \text{ is \textbf{not defective}}, \\
g_2 = t_2 = 1 \implies \lambda_2 \text{ is \textbf{not defective}}.
\end{cases}
\]

\subparagraph{Conclusion:}
Since \( g = t \) for all eigenvalues, the matrix \( A \) has \textbf{no defective eigenvalues}.


\vspace{1.2cm}
\paragraph{Problem 2g:} Determine if the matrix \( A = \begin{bmatrix} 1 & 0 & 2 \\ -1 & 1 & 3 \\ 0 & 0 & 2 \end{bmatrix} \) has defective eigenvalues.

\paragraph{Solution:}

\subparagraph{Find Eigenvalues (Algebraic Multiplicity, \( t \))}
The characteristic polynomial is calculated as:
\[
\det(A - \lambda I) = \begin{vmatrix}
1 - \lambda & 0 & 2 \\
-1 & 1 - \lambda & 3 \\
0 & 0 & 2 - \lambda
\end{vmatrix} = (1 - \lambda) \begin{vmatrix}
1 - \lambda & 3 \\
0 & 2 - \lambda
\end{vmatrix} + 2 \begin{vmatrix}
-1 & 1 - \lambda \\
0 & 0
\end{vmatrix}.
\]
\[
= (1 - \lambda)\big[(1 - \lambda)(2 - \lambda)\big] + 2 \cdot 0 = (1 - \lambda)(\lambda^2 - 3\lambda + 2).
\]
\[
= -(\lambda - 1)^2(\lambda - 2).
\]
Eigenvalues and algebraic multiplicities:
\[
\lambda_1 = 1 \quad (t_1 = 2), \quad \lambda_2 = 2 \quad (t_2 = 1).
\]

\subparagraph{Compute Geometric Multiplicities (\( g \))}
\textbf{For \( \lambda_1 = 1 \):}  
\[
A - I = \begin{bmatrix}
0 & 0 & 2 \\
-1 & 0 & 3 \\
0 & 0 & 1
\end{bmatrix} = \begin{bmatrix}
1 & 0 & -3 \\
0 & 0 & 1 \\
0 & 0 & 0
\end{bmatrix}.
\]
\textbf{For \( \lambda_2 = 2 \):}  
\[
A - 2I = \begin{bmatrix}
-1 & 0 & 2 \\
-1 & -1 & 3 \\
0 & 0 & 0
\end{bmatrix} = \begin{bmatrix}
1 & 0 & -2 \\
0 & 1 & -1 \\
0 & 0 & 0
\end{bmatrix}.
\]

\subparagraph{Check for Defective Eigenvalues}  
An eigenvalue is defective if \( g < t \):
\[
\begin{cases}
\lambda_1 = 1: & g_1 = 1 < t_1 = 2 \implies \textbf{Defective}, \\
\lambda_2 = 2: & g_2 = 1 = t_2 = 1 \implies \textbf{Not Defective}.
\end{cases}
\]

\subparagraph{Conclusion:}  
The matrix \( A \) has \textbf{one defective eigenvalue} (\( \lambda = 1 \)) because its geometric multiplicity (\( g = 1 \)) is less than its algebraic multiplicity (\( t = 2 \)). The eigenvalue \( \lambda = 2 \) is not defective.

\pagebreak

2h. Let $A_{ij}$ be a square upper triangular matrix. Prove that A is only invertible if the
diagonal in the matrix is not zero.

\paragraph{Proof}
Let \( A \) be an \( i \times j \) upper triangular matrix. A matrix is invertible if and only if its determinant is nonzero. 

For an upper triangular matrix, the determinant is the product of its diagonal entries:
\[
\det(A) = a_{11} \cdot a_{22} \cdot \cdots \cdot a_{ij}
\]

\begin{itemize}
    \item \textbf{1st:} 
    If \( A \) is invertible, then \( \det(A) \neq 0 \). This implies:
    \[
    a_{11} \cdot a_{22} \cdot \cdots \cdot a_{nn} \neq 0 \quad \implies \quad a_{ij} \neq 0 \text{ for all } i = 1, 2, \dots, n.
    \]
    Thus, all diagonal entries must be nonzero.
    
    \item \textbf{2nd:} 
    If all diagonal entries \( a_{ii} \) are nonzero, then:
    \[
    \det(A) = a_{11} \cdot a_{22} \cdot \cdots \cdot a_{ij} \neq 0.
    \]
    Since the determinant is nonzero, \( A \) is invertible.
\end{itemize}

\paragraph{Conclusion}
A square upper triangular matrix \( A \) is invertible \textbf{if and only if} all its diagonal entries are nonzero. 
\[
\boxed{\text{\( A \) is invertible } \iff a_{ij} \neq 0 \text{ for all } i = 1, 2, \dots, n.}
\]

\vspace{1.2cm}

2i. Prove if A is a square upper triangular matrix, then the eigenvalues of A are diagonal values

\paragraph*{Solution:}
\begin{enumerate}
    \item An matrix $A \epsilon M_n$ ,
    det(A) = $\lambda_1 * \lambda_2 * ... \lambda_n$
    \item Matrix A is only invertible if and only if its determinant is non-zero.
    \item An eigen values of an nxn upper triangular matrix are its \textbf{diagonal values}
\end{enumerate}

\pagebreak

2j. Define the block matrix B = \(
\begin{bmatrix}
    A & I \\
    C & 0 
\end{bmatrix}
\), where I is an identity matrix and O is a zero matrix. Calculate the $BB^T$.

\paragraph{Solution:}
\[
B^T = \begin{bmatrix}
    A^T & C^T \\
    I & 0 
\end{bmatrix}.
\]
\[
BB^T = \begin{bmatrix}
    A & I \\
    C & 0 
\end{bmatrix}
\begin{bmatrix}
    A^T & C^T \\
    I & 0 
\end{bmatrix}
= \begin{bmatrix}
    AA^T + I & AC^T \\
    CA^T & CC^T 
\end{bmatrix}.
\]

\boxed{BB^T = \begin{bmatrix}
    AA^T + I & AC^T \\
    CA^T & CC^T 
\end{bmatrix}}

\vspace{1.2cm}

\textbf{Prove:} Let \( I \) be the \( n \times n \) identity matrix and \( u \) and \( v \) be two \( n \times 1 \) column vectors. The matrix \( I + uv^T \) is invertible if and only if \( 1 + v^T u \neq 0 \). When it is invertible, its inverse is
\[
(I + uv^T)^{-1} = I - \frac{1}{1 + v^T u} uv^T.
\]

\textbf{What is the Woodbury matrix identity (no proof needed)?}

\paragraph*{Soln:}
The matrix \(I + uv^T\) is invertible if and only if \(1 + v^Tu \neq 0\). When invertible, the inverse can be represented: 
\[
(I + uv^T)^{-1} = I - \frac{1}{1+v^Tu}uv^T
\]


Assume \( I + uv^T \) is invertible. Then:
\[
    (I + uv^T)u = u + u(v^T u) = u(1 + v^T u) = 0
\]
This implies \( u = 0 \) or \( 1 + v^T u = 0 \). If \( u \neq 0 \), then \( 1 + v^T u = 0 \), contradicting the invertibility of \( I + uv^T \). Hence, \( I + uv^T \) is invertible if and only if \( 1 + v^T u \neq 0 \).

\vspace{.5cm}

\textbf{Woodbury Matrix identity}

\[
\boxed{
    (A+UCV)^{-1} = A^{-1} - A^{-1}U(C^{-1} + VA^{-1}U)^{-1}VA^{-1}
}
\]
\end{document}
