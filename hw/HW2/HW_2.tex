\documentclass[12pt,letterpaper]{article}
\usepackage{fullpage}
\usepackage{amsmath, amsfonts, amssymb, amsthm, amscd} % Combined math packages
\usepackage{tikz}     % For drawing matrices
\usepackage{listings, pythontex} % For code snippets
\usepackage{enumerate}
\usepackage{fancyhdr}
\usepackage{mathrsfs}
\usepackage{xcolor}
\usepackage{graphicx}
\usepackage{hyperref}
\usepackage{booktabs}
\usepackage{relsize}

\hypersetup{%
  colorlinks=true,
  linkcolor=blue,
  linkbordercolor={0 0 1}
}

\setlength{\parindent}{0.0in}
\setlength{\parskip}{0.2in} % Increased spacing for readability

%%%%%%%%%%%%%%%%%%%%%%%%%%%%%%%%%%%%%%%%%%%%%%%%%%%%%%%%%%

%%%%%%%%%%%%%%%%%%%%%%%% CHANGE HERE %%%%%%%%%%%%%%%%%%%%
\newcommand\course{STA4724}
\newcommand\semester{Spring 2025}
\newcommand\hwnumber{2}                 % <-- ASSIGNMENT #
\newcommand\NetIDa{Albert Jojo}          % <-- YOUR NAME
\newcommand\NetIDb{5390131}              % <-- STUDENT ID #
%%%%%%%%%%%%%%%%%%%%%%%%%%%%%%%%%%%%%%%%%%%%%%%%%%%%%%%%%%

%%%%%%%%%%%%%%%%% DO NOT CHANGE HERE %%%%%%%%%%%%%%%%%%%%
\pagestyle{fancyplain}
\headheight 35pt
\lhead{\NetIDa \\ \NetIDb} % Combined name and ID in the left header
\chead{\textbf{\Large Homework \hwnumber}}
\rhead{\course \\ \semester}
\lfoot{}
\cfoot{}
\rfoot{\small\thepage}
\headsep 1.5em
%%%%%%%%%%%%%%%%%%%%%%%%%%%%%%%%%%%%%%%%%%%%%%%%%%%%%%%%%%
\begin{document}

\section*{Question 2}
\begin{itemize}
    \item[(a)] The problem of learning an input-output mapping is called \underline{\hspace{3cm}} problem.
    \vspace{0.5cm}
    \\
    \textbf{Answer:} a supervised learning problem.
    
    \item[(b)] A supervised learning problem is called \underline{\hspace{3cm}} if the output variable \(y_t\) is discrete/categorical.
    \vspace{0.5cm}
    \\
    \textbf{Answer:} a classification problem.
    
    \item[(c)] A supervised learning problem is called \underline{\hspace{3cm}} if the output variable \(y_t\) is continuous.
    \vspace{0.5cm}
    \\
    \textbf{Answer:} a regression problem.
    
    \item[(d)] Denote the realized outputs \(y_i\), predicted outputs \(\hat{y}_i\) and features of \(x_i\).
    The best predictive model \textit{h} is the one having the smallest expected loss \(L(h) = E[l(h_{\theta}(x),y)]\) with
    respect to the joint distribution of \(x_i\) and \(y_i\). Since the true joint distribution of \(x_i\) and \(y_i\) is usually
    unknown, the empirical risk is calculated as the sample average of the losses over a set of observed data.
    \vspace{1.0cm}
    \\
    \textbf{Answer:} True.
    
    \item[(e)] Write down the formulas of the following loss functions: Mean Squared Error (MSE), Mean Absolute Error (MAE), and Cross Entropy.
    \vspace{1.0cm}
    \\
    \textbf{Answer:}
    
    \textit{Mean Squared Error (MSE)}
    \[
        \textbf{MSE} = \frac{1}{n}\sum_{i=1}^{n} (y_i - \hat{y}_i)^2
    \]
    
    \textit{Mean Absolute Error (MAE)}
    \[
        \textbf{MAE} = \frac{1}{n}\sum_{i=1}^{n} \left\lvert x_i - y_i \right\rvert 
    \]
    
    \textit{Cross Entropy}
    \[
        \textbf{H(P,Q)} = -\sum_{x \in X} P(x) \log Q(x)
    \]
    
    \item[(f)] Show that the excess risk is bounded for the asymptotic analysis of empirical risk minimization:
    \[
        L(\hat{\theta}) - L(\theta^*) \leq \frac{c}{n} + O(\frac{1}{n})
    \]
    
    \textbf{Answer:}
    \begin{itemize}
        \item The left-hand side, \(L(\hat{\theta}) - L(\theta^*)\), represents the excess risk, which is the difference between the loss of the model trained on the data and the loss of the best possible model (\(\theta^*\)).
        \item The right-hand side, \(\frac{c}{n} + O(\frac{1}{n})\), shows that excess risk decreases as the number of data points \(n\) increases.
        \item This means that with more data, our training process becomes more reliable because the empirical distribution of data (observed data) gets closer to the true data distribution (reality).
        \item Thus, to conclude, no matter how small our dataset is, ERM is guaranteed to find a reasonable solution, but it can be improved with more data that we collect.
    \end{itemize}
    
    \item[(g)] Describe the curse of dimensionality and your understanding of overfitting.
    \vspace{1.0cm}
    \\
    \textbf{Answer:} 
    
    The \textit{curse of dimensionality} refers to the challenges that arise when working with high-dimensional data. As the number of features increases, it causes various challenges:
    \begin{itemize}
        \item \textbf{Data sparsity:} In higher dimensions, meaningful correlations become harder to identify.
        \item \textbf{Higher computational complexity:} More features lead to increased processing requirements.
        \item \textbf{Need for more data:} Higher dimensions require exponentially more data to maintain accuracy, which becomes impractical.
    \end{itemize}
    
    Overfitting, on the other hand, occurs when the model learns patterns that are too specific to the training dataset. This causes:
    \begin{itemize}
        \item \textbf{Poor test performance:} The model is overly specialized in training data and fails to generalize.
        \item \textbf{Complexity vs. simplicity:} The model may learn complex noise instead of simple, meaningful patterns.
        \item \textbf{Symptoms:} Large gaps between training and testing accuracy.
    \end{itemize}
    
    To mitigate overfitting, techniques like PCA or regularization should be used.
    \item[(h)] Gradient boosting can be used to build sequences of increasingly complex additive models. Form
    generalized additive models and describe the algorithm of training base learners based on the pseudo-
    residuals.
    \vspace{1.0cm}
    \\ \textbf{Answer:} Gradient Boosting can be represented by the generalized additive model:

    \[
    g(E[y(x_1, x_2, ..., x_p)]) = \beta_0 + f_1(x_1) + f_2(x_2) + \cdots + f_p(x_p)
    \]
    
    The boosting algorithm initializes with \( f_0(X_T) = 0 \) and updates iteratively as follows:
    
    \begin{enumerate}
        \item Compute the pseudo-residuals:
        \[
        r_{j-1} = -\frac{\partial L(Y, f_{j-1}(X_T) + h)}{\partial h} \Big|_{h=0}
        \]
        where \( L(Y, f) \) is the loss function.
        
        \item Fit a base learner \( h_j^* \) (e.g., decision tree) to approximate the pseudo-residuals:
        \[
        h_j^*(X_T) \approx r_{j-1}
        \]
    
        \item Update the model:
        \[
        f_j(X_T) = f_{j-1}(X_T) + \lambda h_j^*(X_T)
        \]
        where \( \lambda \) is a learning rate parameter.
    
        \item Evaluate the empirical risk of \( f_j(X_T) \) on a validation set.
    
        \item Repeat steps 1–4 until the empirical risk stops decreasing or until the maximum number of iterations \( J \) is reached.
    \end{enumerate}
    
\end{itemize}
\pagebreak
\section*{Question 3}
The code for the 2-layer neural network is given below:
\begin{lstlisting}[language=Python, frame=single, breaklines=true]
    For my 2-layer neural model I used dask to generate big data for my model simulation, instead of using numpy to generate values.

    # Importing the required libraries
    import dask.dataframe as dd
    import dask.array as da
    from dask.distributed import Client
    from dask_ml.metrics import accuracy_score
    from dask_ml.model_selection import train_test_split
 
    
    # Layer class
    class Layer:
        def __init__(self):
            self.input = None
            self.output = None
    
        def __call__(self, x):
            return self.forward(x)
    
        def forward(self, input):
            raise NotImplementedError
    
        def backward(self, up_gradient):
            raise NotImplementedError
    
        def step(self, learning_rate):
            pass
    
    # Loss function class
    class Loss:
        def __init__(self):
            self.prediction = None
            self.target = None
            self.loss = None
    
        def __call__(self, prediction, target):
            return self.forward(prediction, target)
    
        def forward(self, prediction, target):
            raise NotImplementedError
    
        def backward(self):
            raise NotImplementedError
    
    # Multi-Layer Perceptron (MLP) class
    class MLP:
        def __init__(self, layers, loss, learning_rate):
            self.layers = layers
            self.loss_fn = loss
            self.learning_rate = learning_rate
    
        def __call__(self, input):
            return self.forward(input)
    
        def forward(self, input):
            for layer in self.layers:
                input = layer(input)
            return input
    
        def backward(self):
            up_gradient = self.loss_fn.backward()
            for layer in reversed(self.layers):
                up_gradient = layer.backward(up_gradient)
    
        def loss(self, prediction, target):
            return self.loss_fn(prediction, target)
    
        def update(self):
            for layer in self.layers:
                layer.step(self.learning_rate)
    
        def train(self, x_train, y_train, epochs, batch_size):
            losses = []
            for epoch in epochs:
                running_loss = 0
                for i in range(0, x_train.shape[0], batch_size):
                    x_batch = x_train[i : i + batch_size]
                    y_batch = y_train[i : i + batch_size]
    
                    prediction = self.forward(x_batch)
    
                    # Compute Loss and ensure computation
                    loss = self.loss_fn(prediction, y_batch).mean().compute()
                    running_loss += loss
    
                    self.backward()
                    self.update()
    
                running_loss /= (x_train.shape[0] / batch_size)
                losses.append(running_loss)
            return losses
    
    # Linear layer
    class Linear(Layer):
        def __init__(self, input_size, output_size):
            super().__init__()
            self.w = 0.1 * da.random.normal(size=(input_size, output_size))
            self.b = da.zeros((1, output_size))
            self.dw = da.zeros_like(self.w)
            self.db = da.zeros_like(self.b)
    
        def forward(self, input):
            self.input = input
            return da.dot(self.input, self.w) + self.b
    
        def backward(self, up_gradient):
            self.dw = da.dot(self.input.T, up_gradient)
            self.db = da.sum(up_gradient, axis=0, keepdims=True)
            return da.dot(up_gradient, self.w.T)
    
        def step(self, learning_rate):
            self.w -= learning_rate * self.dw
            self.b -= learning_rate * self.db
    
    # ReLU activation function
    class ReLU(Layer):
        def forward(self, input):
            self.input = input
            return da.maximum(0, input)
    
        def backward(self, up_gradient):
            return up_gradient * (self.input > 0)
    
    # CrossEntropy loss
    class CrossEntropy(Loss):
        def forward(self, prediction, target):
            exp_pred = da.exp(prediction - da.max(prediction, axis=1, keepdims=True))
            self.softmax = exp_pred / da.sum(exp_pred, axis=1, keepdims=True)
            self.target = target
            return -da.mean(da.sum(target * da.log(self.softmax + 1e-15), axis=1))
    
        def backward(self):
            return (self.softmax - self.target) / self.target.shape[0]
    
    # Mean Squared Error loss
    class MSE(Loss):
        def forward(self, prediction, target):
            self.prediction = prediction
            self.target = target
            return da.mean((prediction - target) ** 2)
    
        def backward(self):
            return 2 * (self.prediction - self.target) / len(self.target)
    
    # Simulate data using Dask 
    def simulate_data():
        client = Client()  
    
        n_samples = 50000
        n_features = 100
    
        # Generate data 
        X = da.random.random((n_samples, n_features), chunks=(100_000, n_features))
        y = da.random.randint(0, 2, size=(n_samples,), chunks=(100_000,))
    
        # Convert to Dask DataFrame
        X_dd = dd.from_dask_array(X, columns=[f'feature{i}' for i in range(n_features)])
        y_dd = dd.from_dask_array(y, columns=['target'])
    
        # Split into training & test sets (85% train, 15% test)
        X_train, X_test, y_train, y_test = train_test_split(X_dd, y_dd, test_size=0.15, random_state=42)
    
        # Return dask arrays
        return X_train.to_dask_array(lengths=True), X_test.to_dask_array(lengths=True), \
               y_train.to_dask_array(lengths=True), y_test.to_dask_array(lengths=True)
    
    # Create network layers (2-layer neural network)
    layers = [
        Linear(100, 50),
        ReLU(),
        Linear(50, 2)
    ]
    
    # Load data
    x_train, x_test, y_train, y_test = simulate_data()
    
    # Create the model
    model = MLP(layers, CrossEntropy(), learning_rate=0.01)
    
    # Train the model
    losses = model.train(x_train=x_train, y_train=y_train, epochs=30, batch_size=1000)
    
    # Compute accuracy
    y_pred = model(x_test).compute()
    accuracy = accuracy_score(y_test.compute(), np.argmax(y_pred, axis=1))
    
    print(f"Accuracy:{accuracy}")
    
    
\end{lstlisting}
\end{document}
